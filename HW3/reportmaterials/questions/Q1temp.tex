\section{خواص سیستم ها}
شش خاصیت «خطّی بودن»، «تغییرناپذیری با زمان»، «بی حافظگی» ،«وارون پذیری»، «علّیت»، و «پایداری» را برای
هر یک از سیستم های زیر تحقیق کنید. (لازم است برای هر ویژگی از هر سیگنال، استدلال مختصری ارائه دهید.)

\begin{qsolve}
	\textbullet\ $y[n]=e^{x[n]}$
	\begin{qsolve}[]
		\textbf{خطی بودن:} نادرست
		\begin{eqnarray*}
			\letm y_1[n]&=&e^{x_1[n]},\, y_2[n]=e^{x_2[n]},\,
			y_3[n]=y[n]\when_{x[n]=x_1[n]+x_2[n]}\\
			y_3[n]&=&e^{x_1[n]+x_2[n]} \neq e^{x_1[n]} + e^{x_2[n]} = y_1[n]+y_2[n]
		\end{eqnarray*}
		\textbf{تغییر ناپذیری با زمان:} درست
		\begin{eqnarray*}
			\letm y_1[n]&=&e^{x_1[n]},\, y_2[n]=y[n]\when_{x[n]=x_1[n+m]}\\
			y_2[n]&=&e^{x_1[n+m]} = y_1[n+m]
		\end{eqnarray*}
		\textbf{بی حافظگی:} درست
		\begin{eqnarray}\label{eq:1}
			h[n]=y[n]\when_{x[n]=\delta[n]}=e^{\delta[n]}=
			\begin{cases}
				e & n = 0    \\
				1 & n \neq 0
			\end{cases}
		\end{eqnarray}
		فقط زمانی که ورودی داریم خروجی تغییر کرده و هیچ اثری از
		آن در زمان های بعد نیست پس سیستم حافظه ندارد.\\
		\textbf{وارون پذیری}: درست
		$$y[n]=e^{x[n]}\Rightarrow x[n]=\log{(y[n])}$$
		سیستم یک به یک است پس وارون پذیر است.\\
		\textbf{علیت:} درست\\
		همانطور که در معادله \ref*{eq:1} میبینیم سیستم تا
		قبل از ورودی دادن در استراحت است و قبل از آن از ورودی
		اطلاعی ندارد، پس یک سیستم علّی است.\\
		\textbf{پایداری:} درست\\
		همانطور که در معادله \ref*{eq:1} میبینیم، بعد از
		دادن ورودی سیستم به حالت تعادل خود بازگشته و پایدار است.
	\end{qsolve}
\end{qsolve}
\begin{qsolve}[]
	\textbullet\ $y(t)=x(2)+x(t-6)$
	\begin{qsolve}[]
		\textbf{خطی بودن:} درست
		\begin{eqnarray*}
			\letm y_1(t)&=&x_1(2)+x_1(t-6),\, y_2(t)=x_2(2)+x_2(t-6),\,
			y_3(t)=y(t)\when_{x(t)=x_1(t)+x_2(t)}\\
			y_3(t)&=&(x_1(2)+x_2(2))+(x_1(t-6)+x_2(t-6))\\
			&=& x_1(2)+x_1(t-6) + x_2(2)+x_2(t-6)\\
			&=& y_1(t)+y_2(t)
		\end{eqnarray*}
		\textbf{تغییر ناپذیری با زمان:} نادرست
		\begin{eqnarray*}
			\letm y_1(t)&=&x_1(2)+x_1(t-6),\, y_2(t)=y(t)\when_{x(t)=x_1(t+T)}\\
			y_2(t)&=&x_1(2+T)+x_1(t+T-6) \\
			&\neq& y_1(t+T)
		\end{eqnarray*}
		\textbf{بی حافظگی:} درست
		\begin{eqnarray}\label{eq:1}
			h(t)=y(t)\when_{x(t)=\delta(t)}=\delta(2)+\delta(t-6)=\delta(t-6)
			\begin{cases}
				1 & n = 6    \\
				0 & n \neq 6
			\end{cases}
		\end{eqnarray}
		هیچ اثری از ورودی در زمان های بعد نیست پس سیستم حافظه ندارد.\\
		\textbf{وارون پذیری}: نادرست
		سیستم یک به یک است پس وارون پذیر است.\\
		\textbf{علیت:} صحیح\\
		همانطور که در معادله \ref*{eq:1} میبینیم سیستم تا
		قبل از ورودی دادن در استراحت است و قبل از آن از ورودی
		اطلاعی ندارد، پس یک سیستم علّی است.\\
		\textbf{پایداری:} صحیح\\
		همانطور که در معادله \ref*{eq:1} میبینیم، بعد از
		دادن ورودی سیستم به حالت تعادل خود بازگشته و پایدار است.
	\end{qsolve}
	\textbullet\ $y(t)=\frac{\sin(x(t)+2t)}{x(t-1)}$
	\begin{qsolve}[]
		\textbf{خطی بودن:} درست
		\begin{eqnarray*}
			\letm y_1(t)&=&x_1(2)+x_1(t-6),\, y_2(t)=x_2(2)+x_2(t-6),\,
			y_3(t)=y(t)\when_{x(t)=x_1(t)+x_2(t)}\\
			y_3(t)&=&(x_1(2)+x_2(2))+(x_1(t-6)+x_2(t-6))\\
			&=& x_1(2)+x_1(t-6) + x_2(2)+x_2(t-6)\\
			&=& y_1(t)+y_2(t)
		\end{eqnarray*}
		\textbf{تغییر ناپذیری با زمان:} نادرست
		\begin{eqnarray*}
			\letm y_1(t)&=&x_1(2)+x_1(t-6),\, y_2(t)=y(t)\when_{x(t)=x_1(t+T)}\\
			y_2(t)&=&x_1(2+T)+x_1(t+T-6) \\
			&\neq& y_1(t+T)
		\end{eqnarray*}
	\end{qsolve}
\end{qsolve}
\begin{qsolve}[]
	\begin{qsolve}[]
		\textbf{بی حافظگی:} درست
		\begin{eqnarray}\label{eq:1}
			h(t)=y(t)\when_{x(t)=\delta(t)}=\delta(2)+\delta(t-6)=\delta(t-6)
			\begin{cases}
				1 & n = 6    \\
				0 & n \neq 6
			\end{cases}
		\end{eqnarray}
		هیچ اثری از ورودی در زمان های بعد نیست پس سیستم حافظه ندارد.\\
		\textbf{وارون پذیری}: نادرست
		سیستم یک به یک است پس وارون پذیر است.\\
		\textbf{علیت:} صحیح\\
		همانطور که در معادله \ref*{eq:1} میبینیم سیستم تا
		قبل از ورودی دادن در استراحت است و قبل از آن از ورودی
		اطلاعی ندارد، پس یک سیستم علّی است.\\
		\textbf{پایداری:} صحیح\\
		همانطور که در معادله \ref*{eq:1} میبینیم، بعد از
		دادن ورودی سیستم به حالت تعادل خود بازگشته و پایدار است.
	\end{qsolve}
\end{qsolve}